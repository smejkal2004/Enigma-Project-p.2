\documentclass{article}
\usepackage{graphicx} % Required for inserting images

\title{Enigma Project Report}
\author{Jakub Šmejkal, Suweyda Ugas}
\date{December 2025}

\begin{document}

\maketitle

\newpage
\tableofcontents
\newpage

\section{Encrypt function}
\subsection{Logic and Functionality}

The first steps of working on the encrypt function were first to simplify the exercise at hand
into simpler steps, that can be tackled one by one, this process is also known as the top-down method.
The assignment has been broken down into the following steps:

\begin{enumerate}
    \item Find a way to set the initial offset of the Enigma Machine to the offset given in the form of a 3-letter string \textit{rotors}.
    \item Simulate the running of the Enigma Machine with the initial offset and the letters in \textit{message} typed in individually.
    \item Find a way to record the ouput of the enctryption process of each individual letter.
    \item Return a string of the encrypted letters.
\end{enumerate}

\noindent\title{\textbf{Setup}}\newline

\noindent Before we started with writing any new code, we made sure to \textit{import EnigmaModel from EnigmaModel}, which allowed us to use
functions and variables defined in the \textit{EnigmaModel.py} file.
\newline\par\noindent Next, we chose to define the variable \textit{alphabet} as the string consisting of all capital letter in the english alphabet,
as this was used several times as the backbone to many functions and steps in EnigmaModel.py.

\hfill \break
\noindent\title{\textbf{Step 1}}\newline

\noindent We needed to have a way of accessing the rotors' offsets from the EnigmaModel.py file in order to change their initial values from "AAA" or 000 to the values given in \textit{rotors}.
We chose to do that by assigning \textit{EnigmaModel()} to a variable \textit{model}.
\newline\par\noindent Now we could start focusing on the translation of \textit{rotors} into the initial rotor offsets. What 
we needed to do was to turn letters into numbers into offset fx. "ABC" == "012" == 0 and 1 and 2, the important observation here is that \textbf{the index of a given letter in the alphabet is also the offset of the rotor when showing that specific letter}. 
Thanks to that, we could implement the following code:
\newline\par\centerline{model.rotorX.offset = alphabet.index(rotors[X])}
\hfill \break
\par\noindent To put it simply, this finds the offset of a given rotor in the EnigmaModel.py and assigns to it the value of the index 
of the first occurance of a given letter in the alphabet string.

\hfill \break
\noindent\title{\textbf{Step 2}}\newline

\noindent B

\hfill \break
\noindent\title{\textbf{Step 3}}\newline

\noindent B

\hfill \break
\noindent\title{\textbf{Step 4}}\newline

\noindent B

\hfill \break
\subsection{Debugging}

\newpage

\section{Find Rotors function}
\subsection{Functionality and Logic}
\subsection{Debugging}

\end{document}
